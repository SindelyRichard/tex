\documentclass[]{article}
% article, book, %+enter=semmi
\usepackage[magyar]{babel}
\usepackage{t1enc}
\usepackage{xcolor}
\usepackage{graphicx}
\begin{document}
% valami megjeleníthető
\begin{center}
Helló világ
\end{center}
\texttt{A typewriter típus}
és
\textbf{vastag típus}
utána
\textsc{kiskapitális}
\textsl{slanted}
és
\textsf{sans serif.} \\
Az emph parancs így hat a különböző betűtípusokra:
\emph{\textit{italic} \textbf{vastag} \textsc{kiskapitális.}}

Most megnézzük hogy hogy néz ki betűtípusok különböző kombinációi:
\texttt{\textbf{typerwriter és vastag}},
\textsc{\textit{kiskapitális és italic}},
\textsl{\textsf{slanted és sans serif}},
Az emph parancs így hat a kombinált típusra:
\emph{\textit{\textbf{\textsc{italic, vastag és a kiskapitálís}}}}
{Ez itt egy mondat ahol ennek a \large szónak \normalsize nagyobb lesz a betűmérete, ez a \LARGE szó \normalsize meg csupa nagybetűvel lesz írva.}

Ez a \reflectbox{\framebox[1cm]{szó}} függőlegesen tükrözve van a tengelye körül és be is  van keretezve.
Ez a \scalebox{2.0}[-1]{szó} tükrözve van vízszintes tengely körül és vízszintesen nyújtva is van. 
Ez a \rotatebox{90}{szó} el van forgatva 90 fokkal, ez a \rotatebox{270}{szó} pedig 270-el.\\
\colorbox{black}{\textcolor{white}{Ennek a szövegnek fekete a háttere és a betűk színe fehér.}}\\
\fcolorbox{red}{white}{\textcolor{red}{Ez a szöveg pedig piros keretes és a betűszín is piros.}}
\end{document}