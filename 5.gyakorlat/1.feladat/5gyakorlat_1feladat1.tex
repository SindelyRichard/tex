\documentclass[aspectratio=169,12pt,xcolor={table}]{beamer}
\usepackage{verbatim}
\usepackage{graphicx}
\usepackage{hyperref}
\usepackage{hulipsum}
\usepackage{enumerate}

\author{Sindely Richárd}
\title{5.gyakorlat}
\date{2023}

\usetheme{Copenhagen}

\begin{document}
\frame{\titlepage}
\section{Első dia}
\frame{\tableofcontents[currentsection]}
\begin{frame}{Cím}{Alcím}
\subsection{Kép,lista}
\begin{columns}[c]
\begin{column}{.5\linewidth}
\begin{enumerate}
\item<1> Egy \transduration<1>{2}
\item<2> Kettő \transduration<2>{2}
\end{enumerate}
\begin{itemize}
\item Egy
\item Kettő
\end{itemize}
\end{column}
\begin{column}{.5\linewidth}
\begin{figure}
\includegraphics<1>[height=5cm,width=5cm]{D:/Uni/Tex/5.gyakorlat/1.feladat/szepia.jpg}\transdissolve<1>
\includegraphics<2>[height=5cm,width=5cm]{D:/Uni/Tex/5.gyakorlat/1.feladat/szines.jpg}\transdissolve<2>
\onslide<1>{\caption{szepia}}\transdissolve<1>
\onslide<2>{\caption{szines}}\transdissolve<2>
\end{figure}
\end{column}
\end{columns}
\end{frame}
\section{Második dia}
\frame{\tableofcontents[currentsection]}
\begin{frame}[allowframebreaks]{Másik cím}{Másik alcím}
Ez másik egy diasor.
\hulipsum[1-3]
\end{frame}
\subsection{blockok}
\begin{frame}{Blockok}
\begin{block}{}
Block tartalma
\end{block}
\begin{exampleblock}{Exampleblock címe}
Exampleblock tartalma
\end{exampleblock}
\begin{alertblock}{Alertblock címe} \pause
Alertblock tartalma
\end{alertblock}
\begin{theorem}
$\sum_{k=1}^n \frac{1}{k}$
\end{theorem}
\onslide<3>{
\begin{proof}
Ez egy bizonyítás
\end{proof}}
\end{frame}
\begin{frame}[fragile,allowframebreaks]{Még egy cím}{Még egy alcím}
Ez megint egy másik diasor.
\begin{verbatim}
\begin{frame}{Cím}{Alcím}%
Ez egy diasor.%
\end{frame}%
\end{verbatim}
\begin{semiverbatim}
\\begin\{\alert{frame}\}\{\textcolor{blue}{cím}\}
\end{semiverbatim}
\end{frame}
\end{document}