\documentclass[12pt]{article}
\usepackage[english,magyar]{babel}
\usepackage{t1enc}
\usepackage{blindtext}
\usepackage{lipsum}
\usepackage{hulipsum}
\usepackage{xcolor}
\usepackage{graphicx}
\usepackage[inline]{enumitem}
\usepackage[headheight=12pt]{geometry}
\usepackage{multicol}
\usepackage{fancyhdr}
\pagestyle{fancy}
\fancyhf{}
\fancypagestyle{plain}{%
\fancyhf{}%
\fancyhf[RF]{\thepage}%
}
\fancyhf[RHO]{\thepage}
\fancyhf[CFO,CFE]{Miskolci Egyetem}
\title{Cím}
\author{Sindely Richárd}
\renewcommand{\sectionmark}[1]{\markboth{}{\thesection. #1}}
\renewcommand{\footrulewidth}{0.4pt}
\fancyhf[LHE]{\nouppercase{\leftmark}}
\fancyhf[LHO]{\rightmark}
\begin{document}
\maketitle
\selectlanguage{magyar}
\frenchspacing
\noindent\setlength{\parskip}{12pt}\hulipsum[1]
\today
\begin{flushright}
{\foreignlanguage{latin}{\lipsum[1]}}
\end{flushright}
{
\selectlanguage{english}\linespread{1.6}\selectfont\blindtext[1]
\today
}
\newpage
\section{Cím}
\hulipsum[3]
\newpage
\subsection{Alcím}
\hulipsum[1]
\\
\begin{itemize*}[itemjoin*={\hspace{1em}és },label=\$]
\item Első elem
\item Második elem
\item Harmadik elem
\end{itemize*}
\begin{enumerate}
\item Elem
\item Elem
\begin{enumerate}[label*=.\arabic*]
\item elem
\begin{enumerate}
\item elem
\begin{enumerate}
\item elem
\end{enumerate}
\end{enumerate}
\end{enumerate}
\end{enumerate}
\newpage
\section{Saját lista}
\newlist{myenum}{enumerate}{5}
\setlist[myenum]{label=(\arabic*)}
\newcounter{mycounter}
\setcounter{mycounter}{1}
\begin{myenum}
\item[\label=(\themycounter)] elem
\stepcounter{mycounter}
\item[\label=(\$)] elem
\stepcounter{mycounter}
\item[\label=(\themycounter)] elem
\stepcounter{mycounter}
\begin{myenum}
\item elem
\begin{myenum}
\item elem
\begin{myenum}
\item elem
\begin{myenum}
\item elem
\end{myenum}
\end{myenum}
\end{myenum}
\end{myenum}
\end{myenum}
\hulipsum[2]
\begin{myenum}
\item[\label=(\themycounter)] elem
\stepcounter{mycounter}
\item[\label=(\themycounter)] elem
\stepcounter{mycounter}
\end{myenum}
\newpage
\subsection{Leíró lista}
\begin{description}[style=nextline]
\item[] \hulipsum[1]
\item[Slanted] \hulipsum[2]
\begin{description}
\item[Hosszú címke] \hulipsum[2]
\end{description}
\end{description}
\end{document}